\documentclass[a4paper,landscape]{tarticle}
\usepackage[landscape]{geometry}

% %% 高さの設定
% \setlength{\textheight}{\paperheight}   % ひとまず紙面を本文領域に
% \setlength{\topmargin}{-5.4truemm}      % 上の余白を20mm(=1inch-5.4mm)に
% \addtolength{\topmargin}{-\headheight}  % 
% \addtolength{\topmargin}{-\headsep}     % ヘッダの分だけ本文領域を移動させる
% \addtolength{\textheight}{-40truemm}    % 下の余白も20mmに
% %% 幅の設定
% \setlength{\textwidth}{\paperwidth}     % ひとまず紙面を本文領域に
% \setlength{\oddsidemargin}{-5.4truemm}  % 左の余白を20mm(=1inch-5.4mm)に
% \setlength{\evensidemargin}{-5.4truemm} % 
% \addtolength{\textwidth}{-40truemm}     % 右の余白も20mmに

\setlength{\textheight}{244truemm}
\setlength{\headheight}{0pt}
\setlength{\headsep}{25truemm}
\setlength{\footskip}{15truemm}
\addtolength{\topmargin}{-1truein}


\pagestyle{empty}
\setlength{\columnseprule}{0.3pt}
\begin{document}

近現代文学野の中で「純文学」として、夏目漱石のこゝろ、「歴史・時代小説」として、
樋口一葉のたけくらべを読んだ。 \par
まずは、夏目漱石のこゝろについてまとめる。
この作品は、時系列として明治時代の中であり、
「先生と私」、「両親と私」
、「 \\ 先生と遺書」の三章から成り立っている。

% 第1章
第一章「先生と私」では、人生の生きがいを見つけるために上京した私が、
自分を特別な存在だと思い、生活をしていて、あ \\ る時鎌倉の海岸で先生と会う。
そこで、独特な雰囲気を持つ先生に惹かれ人生の教訓をこの人から
学びたいと思うようになる。 \\ 
先生は無職で、親の財産で妻と二人で生活しており毎月1回
友人の墓参りをしており、この友人が亡くなる前の先生は、
明るく \\ 頼り甲斐のある人物であると妻が言うシーンがあった。
友人が亡くなってからは、人を信頼しない孤独な生活をするようになり \\ 
そのことで私の先生に対する興味は増していったのだ。
私が大学を卒業すると、先生に近況を報告するために家を訪れた時、
先 \\ 生は私に対して、「親の財産が少しでもあるなら
今のうちに始末しておきなさい」を助言した。

% 第2章
第二章「両親と私」では、父の病態の悪化の応報を受け
一度実家に帰省するが、父は元気だった。
しかし、数日後に天皇陛下 \\ 崩御のお知らせを聞くと
たちまち病状が悪化し、衰弱していった。
そこで父は「明治と共になら本望」と言った。
そんな時に、 \\ 先生から
「この手紙があなたの手に落ちている頃には、私は死んでいる」
と書かれた大量の分厚い手紙が届き、父親の病状にも \\ 関わらずに
先生の元を訪れるために先生の元へ向かった。
その道中で手紙を読み進めるが内容は凄まじいものであった。

% 第3章
第三章「先生と遺書」では、その手紙には先生の遺書が書かれていた。
大学在学中に両親を失った先生は、叔父に全ての遺産 \\ を食いつぶされ、
そこから人間不信に陥ったのである。
その後、下宿を始め、大家の娘であった現妻と出会った。
次第に恋をし \\ た。
しかし、友人Kが貧しい生活をしていたので救うために
友人Kと同棲をした。
その最中に、Kが先生に娘が好きだと言う \\ が、
Kに娘を奪われることを危惧した先生は
「精神的に向上心のない者は馬鹿だ」と言った。
焦った挙句、すぐに母親に「娘さ \\ んを下さい」と懇願し、
結婚に至った。
しかし、結婚の応報を聞いた友人Kはショックを受け自殺したが、
遺書には先生への恨 \\ みは書かれていなかった。
先生は、Kの自殺の経緯を知る唯一の人物として、ただ一人苦しんでいたが、
明治天皇崩御の知らせ \\ を聞いて、
「明治の精神と共に」逝くべく、遺書を書くことに決めたと、手紙に書かれていた。
手紙の末文には、
「この打ちあけ \\ られた私の秘密はすべてあなたのこころにしまっていてください」と
書かれていた。

% まとめ
この作品を読み、ちょうど明治時代は開国した時代で、海外の文化が
徐々に浸透してきた時代であった。その文化の代表的な \\ ものに、
古典的な日本の精神には無かった価値観であるエゴイズムがあり、
それは、今回であれば先生が友人Kの自殺へ抱い \\ ていた感情
である。
新しい文化が輸入された時に著者の夏目漱石は、この新しい
エゴイズムは先生のように苦しみを背負う人間 \\ を増やす原因となると
考えたのかもしれない。

%  歴史小説
次に、樋口一葉のたけくらべについてまとめる。
この作品も明治時代に書かれており、
主人公は、大人になると
遊女になるこ \\ とが決まっている吉原に住む十四歳の女の子である美登利と、
将来は僧侶になるまじめな信如である。
子供の頃、美登利は正太 \\ 郎という少年とよく遊んでいたが、実はこの
信如への好意を抱いており、ある日の運動会で木の根につまづいた
信如を見た美登 \\ 利はハンカチを信如へ渡そうとした。
それを見ていた周りの同級生が、二人をからかったことにより、
信如は噂になることを嫌 \\ がって美登利を無視し始めたことから、
心はすれ違い始め、お互い好意を持ちながらも遠ざかっていってしまった。
そんな時に \\ 、美登利が髪を島田髪に変えられ、吉原で遊女になっていくと
気づいていき、困惑していき、誰とも遊ばずに家で引きこもり
に \\ なってしまった。
そんな時に、信如が吉原から離れた仏学校へ転校する前日に、美登利の
家に水仙の造花を投げ入れ、美登利は \\ 誰が投げたのかわからなかったが、
懐古の気持ちを思い出し、それを部屋に飾った。という話である。

% まとめ
この作品は、美登利と信如の心理描写が絶妙に現れており、恋物語として
述べられている。「源氏物語」は光源氏が多くの恋 \\ 人を作る過程で、
本当の自分を見つけるという話だが、
対照的に美登利と信如の二人の心理描写を描いている。
また、明治時代 \\ の作品であるが故、日本文学の本流である源氏物語とは違い、
あまり身分が中心となった描写ではなく、時代相応の淡い恋物語 \\ が
描かれていた印象を持った。

% まとめ
これら二つの夏目漱石のこゝろ、樋口一葉のたけくらべという作品の関係は、
同じ明治時代中の作品であるが、時代の捉え方 \\ の視点が異っていると感じた。
日本文学の本流として、ふさわしい作品であると思った。



\end{document}
