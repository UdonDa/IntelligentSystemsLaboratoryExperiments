\documentclass[twocolumn, 10pt,a4j]{jsarticle}
\usepackage{here}
\usepackage{amsmath}
\usepackage[dvipdfmx]{graphicx}
\usepackage{url}
% プリアンブル
\title{\vspace{-2.5cm}人間機械システム期末レポート}
\author{1610581 堀田 大地}
\date{2018/8/9}
\begin{document}
\maketitle{}
\section{実行の淵と評価の淵について、冷蔵庫の庫内温度の調整の例を用いて説明せよ。}
% 1
  冷蔵庫の冷蔵室の温度はそのままで、冷凍室の温度を下げたいとする。
  実行として、
  \begin{enumerate}
    \item 意図形成 \\
      冷凍室の温度調整つまみを強に回したい。
    \item 行動決定 \\
      冷凍室の扉をあけて、つまみを回して、扉を閉めたい。
    \item 実行 \\
      行動決定を実際に行う。
  \end{enumerate}
  評価として、
  \begin{enumerate}
    \item 知覚 \\
      冷凍室の温度計を見て、冷蔵室内の野菜の状態を調べる
    \item 解釈 \\
      冷凍室の温度が望み通りに下がったことを知り、冷蔵室
      の温度が意に反して下がってしまったことを知る。
    \item 評価 \\
      なんで冷蔵室も下がってしまったの?
  \end{enumerate}
  がある。
  ここでの、「実行の淵」は、目的から実行までであり、温度を調整する方法がわからなかったり、冷凍室内の
  温度調整つまみが見当たらないことであり、「評価の淵」は、実行後の知覚から
  評価までであり、現在の冷蔵室内の温度がよくわからなかったりすることである。
  \par 一般に実行の淵と評価の淵をともに小さくすることが望ましい。

\section{調整法、極限法、恒常法について説明せよ。また、閾値を最も客観的に求めることが
できる方法はどれかを答えよ。}
  \subsection{調整法}
    値を連続的に動かして求めたい値を求めるん的方法である。値が増加中に
    求めたい値を超えてしまった時には、減少させて上下調整して求める方法。
    この方法を繰り返して値を求める。しかし、実験的手続きが被験者に
    全てわかってしまうので、意図的修正を受けやすい。
  \subsection{極限法}
    値が弱いものから強いものへ、途中で被験者の反応が変化したら一旦ストップし、
    次に逆方向へと段階的に下げ、被験者の反応が変化したらストップする。
    これらを交互に繰り返し、反応変化点の平均をとる。しかし、刺激の変化方向が
    被験者に知られているので、被験者の意図的修正を受ける場合がある。
  \subsection{恒常法}
    予め決めておいた刺激条件をランダムな順番で出し、被験者は
    刺激が知覚できたかどうかを答える。各刺激条件が同じ回数だけ
    提示されるまで繰り返す。各刺激条件において被験者が
    シケきを知覚できた確率を計算し、知覚確率が$50\%$になる
    刺激条件を求める。しかし、実験の所要時間が長く、被験者の
    疲労や単調感の影響を受けやすい。
  最も客観的に求められるのは恒常法である。
    
\section{名義尺度、順序尺度、間隔尺度、比例尺度について、それぞれの例を最低ひとつずつ
挙げよ。}
  \subsection{名義尺度}
    単に区別するために用いる尺度であり、例えば電話番号がある。
  \subsection{順序尺度}
    大小関係に意味があるけれども差や比には意味はない尺度であり、
    例として地震の大きさがある。
  \subsection{間隔尺度}
    大小関係に加えて差にも意味があるが比には意味がない尺度であり、
    例としてセルシウス温度がある。
  \subsection{比率尺度}
    大小関係にも差、比にも意味がある尺度であり、
    例として年齢がある。

\section{予測}
\begin{table}[H]
  \centering
  \caption{天気予報士の過去 100 日間の晴/雨の予測の結果}
  \label{my-label}
  \footnotesize
  \begin{tabular}{lll}
  & 実際は晴れだった & 実際は雨だった  \\ \hline
  晴れと予測した& 45日& 15日\\
  雨と予測した& 10日 &30日
  \end{tabular}
\end{table}
\begin{enumerate}
  \item この天気予報士の正答率 \\
    $正答率 = \frac{45 + 30}{100} * 100 = 75\%$
  \item この天気予報士の$d^{'}$ \\
    $d^{'} = 1.47$
\end{enumerate}


\section{究極のヒューマン・インターフェース}
脳で思ったことを全部やってくれて、遠隔操作できて、フィードバックまで何も
しなくてもできるものが究極のヒューマン・インターフェースだと思う。
例えば、「夜ご飯は寿司が食べたい」と思うだけで目の前に寿司を持って来てくれる
なんでもロボットがあれば、究極である。

\section{授業の感想}
1言でまとめると、楽しかった。なぜ楽しいのか。それは先生が一番楽しそうに
授業をしていたからである。他の授業と比較した結果そう思った。
楽しかったです。
\end{document}