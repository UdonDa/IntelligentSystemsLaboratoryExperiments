\documentclass[twocolumn, 10pt,a4j]{jsarticle}
\usepackage{amsmath}
\usepackage[dvipdfmx]{graphicx}
\usepackage{url}
% プリアンブル
\title{\vspace{-2.5cm}11. 自然および強制対流熱伝達}
\author{1610581 堀田 大地}
\date{2018/6/28}
\begin{document}
\maketitle{}
\section{目的}
% 目的
  工業製品は適切な温度状態に保たれなければならない。例えば、コンピュータのCPUは
  稼働中に多量の熱を発するため、うまく放熱されない場合には温度が上昇し続けて
  物理的に演算のできない状態に陥る。よって、効率よく放熱処理を施すことは
  必須である。本実験では熱移動の基本形式の一つである対流熱伝達について理解を
  深める。
\section{原理}
% 原理
    {\bf 自然対流熱伝達}\ 空気や水などの流体ないに温度差が生じると熱膨張による密度差に
  より流体に運動が発生する現象を自然対流と呼び、
  静止した空気中に王恩物体を設置すると壁
  近傍に熱せられた空気の層、つまり温度境界層が生じ、浮力によって層内の空気が
  上方に流動して熱を運んで行く伝熱形式である。
    \par{\bf 強制対流熱伝達}\ 外部からの仕事によって発生する流体の運動を強制対流と呼び、
  このときに行われる熱移動である。
  \\[1ex]\par
    物体の壁面温度を$T_{w}[K]$、流体の壁より十分離れた位置における温度を$T_{0}[K]$、
  面積$A[m^{2}]$の物体表面より単位時間当たりに放出される熱量を$Q[W]$、熱伝達率
  を$h[W/(m^{2}K)]$とすると、
  $Q$は(1)で表せられる。また、$h$の値は対流の種類や強さ、流体の種類によって変化する。
  \begin{equation}
    Q = hA(T_{w} - T_{0})
  \end{equation}

  自然対流による熱伝達率はかなり小さいので、自然対流熱伝達によって放熱を
  行う場合には、高温側壁面にフィンを設けて放熱量の増大を測ることが多い。
  まず、フィンの設置によってどの程度$Q$が増加するかを単純な一枚のフィンの
  場合について考察する。
    \par 定常状態の条件下でフィンの根元を$x$軸の原点にとり、任意断面$x$における
  微小区間$dx$に対する熱収支を考慮すると(2)が導かれる。
    \begin{center}
      \begin{equation}
        (d/dx)[kS・dT / dx]dx - hRdx(T - T_{0}) = 0
      \end{equation}
      \begin{flushleft}
        \ \ \ \ $k$ : フィンの熱伝導率\ ($k$=428\ W/(mK) : 銅) \\
        \ \ \ \ $R$ : フィンのタンメンの接触長\ $2 \times (b+L)$ \\
        \ \ \ \ $S$ : フィン断面積\ $b \times K$ \\
        \ \ \ \ $T$ : $x=x$におけるフィン温度 \\
        \ \ \ \ $T_{0}$ : 周囲の空気温度 \\
        \ \ \ \ $h$ : フィン表面の熱伝達率  
      \end{flushleft}
    \end{center}
    \par $k, S, T_{0}, h = const$とすると、(2)はTに関する二階の常微分方程式(3)
    となり、解は、$T=T_{w}\ at\ x = 0$および、$dT/dx = 0\ at\ x = H$なる
    境界条件により(4)となる。
      \begin{equation}
        d^{2}T / dx^{2} - (hR / kS)(T - T_{0})=0
      \end{equation}
      \begin{equation}
        (T - T_{0}) / (T_{w} - T_{0}) = coshB(H - x) / cosh BH
      \end{equation}
      \begin{equation}
        B = (hR / kS)^{1/2}
      \end{equation}
    
    \par (4)で得られた温度分布$T=T(x)$により、一枚のフィンからの放熱量$Q$は(6)
    で与えられる。
    \begin{eqnarray}
      %(6)
      Q &=& \int_H^0 hR(T - T_{0})dx \nonumber \\
        &=& (hRkS)^{1/2} (T_{w} - T_{0})tanh BH
    \end{eqnarray}
      \par また、$Q$は(7)で定義される基盤からフィンに流入する熱量と等しい。
    \begin{eqnarray}
      %(7)
      Q = -kS[dT / dx]_{x=0}
    \end{eqnarray}
      \par 次に、フィンを付けたことによる放熱量の増加分について検討する。
    フィンがない場合(接触面積\ =\ $S$)の放熱量を$Q_{0}$とし、
    フィンを付けた場合(接触面積\ =\ $RH$)の放熱量$Q$との比を
    $\epsilon$とすると(8)が成り立つ。
    \begin{eqnarray}
      %(8)
      \epsilon = Q/Q_{0} = (kR / hS)^{1/2}tanh BH
    \end{eqnarray}

      \par 以上の結果より、$N$枚のフィンによる放熱量$Q_{N}$は
    (6)による$Q$を用いて(9)で与えられる。ただし、$A_{w}$は
    フィンの表面以外での流体との接触面積$PL$であり、
    フィンと空気との接触面積$A_{f}=2HL$に比べて省略できるものと考える。
    ここで(6)と(9)は、$Q_{N}$を大きくするためには、
    フィンの熱伝導率$k$、枚数$N$、高さ$H$を大きくし、
    厚さ$b$を小さくすればよいことを示している。
    \begin{eqnarray}
      Q_{N} = N[hA_{w} (T_{w} - T_{0}) + Q] \cong NQ
    \end{eqnarray}

\section{実験装置および実験方法}
% 3
    \par 銅性基盤の寸法は$40.0 \times 42.0 mm$であり、指定された厚さ$b$、
  ピッチ$P$、高さ$H$および枚数$N$のフィンに対して実験を行なった。
    \par まずフィン基盤を水平に設置し(この状態での回転角$\theta$を0にとった)、
  加熱量一定の条件下でフィンを加熱する。フィン基盤の平行温度が$100 - 150 ^\circ C$
  となるように、ヒータの出力を調整してフィン基盤温度の時間的変化を調べる。
  基盤温度の平均値$T_{wm}$と雰囲気の温度$T_{0}$との差が安定するまで測定を行なった。
  次にフィン基盤の回転角$\theta$を$45^\circ$変化させて、
  $\theta = 45^\circ, 90^\circ, 135^\circ, 180^\circ$において同様の測定を
  繰り返した。
    \par 次に、$\theta = 180^\circ$の状態において、逆風機の電源を入れ、
    一定流量の空気を流し自然対流時と同様に$T_{wm}$と$T_{0}$との差が安定するまで
  測定を行なった。
    \par 実験は加熱量一定の条件下行うので、フィン基盤回転角$\theta$がある値での、
  定常状態における基盤平均温度と雰囲気温度との差は(10)を満たす。
    \begin{eqnarray}
      IE = hA_{f}(T_{wm} - T_{0})
    \end{eqnarray}
  よって、フィン表面の平均熱伝達率$h[W/(m^{2}K)]$を(11)より求めることができた。
    \begin{eqnarray}
      h = IE / [A_{f}(T_{wm} - T_{0})]
    \end{eqnarray}

\subsection{原理-2}a
\section{方法}
% 方法
\subsection{準備}
\begin{enumerate}
\item a
\item b
\item c
\item d
\end{enumerate}
\section{結果}
% 結果
\begin{thebibliography}{3}
\bibitem{}電気通信大学 共通教育部自然科学部会(物理) :基礎科学実験A p73-79
\end{thebibliography}
\end{document}