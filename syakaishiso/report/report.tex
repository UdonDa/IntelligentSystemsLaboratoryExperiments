\documentclass[11pt,a4j]{jsarticle}
\usepackage{amsmath}
\usepackage[dvipdfmx]{graphicx}
\usepackage{url}
% プリアンブル
\title{\vspace{-2.5cm}社会思想 レポート}
\author{1610581 堀田 大地}
\date{2018/8/1}
\begin{document}
\maketitle{}
\section{ルソー・マルクス・ニーチェの比較}
\par ルソーとは、歴史的背景としてホッブズのリヴァイアザンやロックの市民
政府論があり、前者は万人の万人に対する闘争状態である自然状態があり、この状態では互いに殺し合い
絶滅してしまうので互いに契約を交わし第三者似た対して自然権を全面的に登城し、すべてに自然権を
集約して成立するものであり、後者は、自然の中で生きる人間は自然に労働や成果を取得し、安定した秩序
の元で生きているが、次第に貨幣が使われるようになり、所有物が蓄積されるような時代になると、
不平等が生じ、犯罪が起きるようになった。これに対処するするために、新たな社会契約が必要となり、
自然権の一部を譲渡する形で政府を設立し、この政府は自然状態における不都合を解消するという明確な
目的を持った結社とすることである。しかし、一部の自然を譲渡した人間には抵抗権が残され、それは
契約違反の政府に対し、別の政府に取り替えようとする行為であり、これを革命とした。
これら2点の背景では十分にと問われなかった不平等の問題が現実的であり政治的問題として
浮上してきて、歴史概念の導入や自然概念の刷新を行った人である。具体的には、「社会契約論」や「エミール」
などの書物を出版した。「社会契約論」において、不平等な問題をいかに回復させて行くのかを追求しており、
個人の私的利益を追求する思想(特殊思想)・特殊石の総和である全体意思は不平等であり、各人が公共の
利益を求める思想(一般思想)は、自由と平等を保証すために全員がそれに従うことによって自由が保証されると
考え、特殊意思や全体意思も満たすことになると追求した。コンセプトとして、力(専制的権力)は社会の
潜在的戦争状態を解決せず、これは社会契約によってこそ解決されるが、その契約は一般意志に基づかねば
ならばい。一般意志とは、各人の自由意志に基づく討論を通じて、自分にも社会全体にも共通する問題意識と判断
とが練り上げられ、発見されることである。このような平等主義思想がルソー没後のフランス人権宣言
に大きな影響を与えた。
\par マルクスとは、ドイツの哲学者、経済学者、社会学者、革命家で、西欧思想を独自に体系化した
人物である。ルソーが民衆に対して平等を訴えかけたあと、カントやヘーゲルがそのような体系の構築を行った。
後にマルクスが活躍しだしたのである。特に有名な物として資本論や地代論がある。資本論とは、

ある生産力Aに対しその時代の社会は生産関係Aで対応する。それを基本的な条件として方や支配関係など
上部構造Aがその上に成立していくが、そうした社会環境の中で、人間たちはいくつかの階級に分かれていく。
だが、階級の相互の間には協調関係が成立するとは限らず、必ず利害・対立関係が潜在している。その中で
その相互の間の関係を商品というものの存在から分析することで解明していったという著書である。
地代論とは、資本論の白眉とされ、繰り返されるバブル経済つまり資本主義の隆盛の発端は、常に地価の
高騰から生じており、土地も労働力も決して商品化されていなかったが、それらひとつひとつが商品化されて
いき、万物の商品化となっていく。この過程で時間が取引の対象となっていくことを論じた著書である。
しかし、これらの著書でも理論上の不備として様々な課題が残されていた。労働の高度化、生産中心の
社会から消費中心の社会への移行、大衆化とファシズムの問題など、20世紀の出来事を説明できるかどうか
という点、労働者家族の内部を支配する家父長制や性別役割分業、あるいは生産に必要な資源の伐採や
不用品の廃棄がもたらすエコロジー問題への意識が低いのではといったことであり、これらは建設的な
課題であった。

\par ニーチェとは、上記のマルクスの論じたもので課題として捨てられていたものから議論を開始している。
また、ニーチェの有名作品として道徳の系譜がある。これは、善悪の判断が生まれてきた理由や善悪
の判断そのものの価値を明らかにするために、道徳の価値を遡って系譜学的に考察していくという著書で
ある。構成としては、第一論文では道徳成立を促す下からの力について、第二論文では道徳成立を要請する上からの力
について、第三論文では両者を結び合せるものについて語られている。諸系譜が1点に合流するところで
道徳が成立している。ニーチェよりも先代のルソーでは、道徳性は彼の描く自然状態に根すと位置付けられていたが、
このニーチェの議論は、ルソー流の本質主義を批判するものである。
\par ルソー・マルクス・ニーチェの3者の比較となるが、これら3人は歴史的に多くの人々を熱狂させた思想を提案した
人々であるが、中には「悪の思想」と捉え嫌う人もいた。実際に、3人全員の思想は悪用されたケースもある。
ルソーではフランス革命のジャコバン党独裁である。これは、ルソーは「一般意志」は無謬だと言ったが、彼が想定して
いたのは都市国家レベルの共同体であり、一般意志は討議を通じて練り上げられるものだったはずだが、このケースでは
違うものとして使われた。また、マルクスでは、共産主義国家や収容所群島である。これは、マルクスは「プロレタリア」
独裁が必要と考えたが、それは革命の過程における過渡的形態だったはずで彼の共産主義のイメージは
「労働組合が発展した」何かであった。ニーチェでは、ナチズムであった。ニーチェの妹であるエリザベートが
遺稿を勝手に改ざんし、原稿をナチスにうった。このニーチェの「力への意志」説、絶対的な真理などどこにもない
という議論は、ナチスが伝統的な人論を踏みにじるのに利用された。

\section{現代社会への示唆}
\par 現代社会への示唆として、ニーチェのにニヒリズムを考える。ニヒリズムとは、世の中にあるものは全て
無意味であり、意味があるような気がするものは妄想であるという考え方である。
マルクスは、特に「資本論」から影響を受けた人も多いであろう。マルクスの捉えた資本主義の本質である資本が
無限に自己増殖する価値運動であるということが、現代社会への示唆であるだろう。
\end{document}