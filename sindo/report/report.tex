\documentclass[twocolumn, 10pt,a4j]{jsarticle}
\usepackage{amsmath}
\usepackage[dvipdfmx]{graphicx}
\usepackage{url}
% プリアンブル
\title{\vspace{-2.5cm}振動の計測と制御}
\author{1610581 堀田 大地}
\date{2018/5/17}
\begin{document}
\maketitle{}
\section{目的}
% 目的
    機械システムの特性や性能を評価する方法の1つに動特性があり,機械の静的
  なひずみや熱膨張などの比較的ゆっくりした特性とは異なり,時間的に応答が早い現象である.
    \par 本実験では,振動現象のモデル化と数学的取り扱い方法,加速度センサを用いた振動の
  計測方法,機械構造の振動挙動,FFTアナライザを用いた振動解析方法を理解することを目的とする.

\section{圧電素子式の加速度センサの特性}
  \subsection{目的}
      入力を加速度センサの振動,出力を得られた信号としたときに,この入力と出力の関係を
    導出することを目的とする.
  \subsection{実験装置,計測機器および解析装置}
    % 問4参照     
  \subsection{実験手順}
    \begin{enumerate}
      \item 加振器の振動ヘッドに加速度センサーを取り付け,振動を加えた.
      \item 加速度センサーの出力値を増幅器で増幅し,FFTアナライザで波形を観測した.
      このとき,$x$軸のスケールは0.125sec,$y$軸のスケールは0.4472 Vとした.
      \item 各周波数毎に画面を静止させ,波形を記録した.
      \item 加振器の周波数を20,40,60,80,100Hzに変化させ,波形と$dx, dy$の値を観測し記録した.
    \end{enumerate}
  \subsection{実験結果}
      % zikken1.tex
      % 問6参照
      周波数が大きくなるごとに波の周期が短くなっていった.
  \subsection{考察}
    \begin{enumerate}
      \item 入力信号の周波数が変化すると,出力電圧にはどのような関係や特徴があったか \\
          入力信号(周波数)と出力信号(電圧)は,40Hzのとき,-35.329 Vと値は一番大きくなったが,
        それ以降の周波数では減衰していった.
      \item 圧電素子の与えられた力と発生する電圧の変換係数はいくつか.
          \begin{equation}
            F = -mhω^{2} = - (1.5 \times 10^{-3}) \times (0.25 \times 10^{-3})^{2} \times (2 \times \pi \times 20)^{2}  = -5.92 \times 10^{-3}  \\
            A = \frac{F}{V} = \frac{-5.92 \times 10^(-3)}{-14.744} = 0.401 NV^{-1}
          \end{equation}
    \end{enumerate}
  \section{1自由度系の強制振動系の挙動}
      \subsection{目的}
          簡単なばね質量系の振動を計測し,共振周波数からばね係数を推定し,静的荷重による変形から
          求めたバネ係数との比較や考察を行う.
      \subsection{実験装置,計測機器および解析装置}
          % fig.pdf
      \subsection{実験手順}
          \begin{enumerate}
            \item 振器の振動ヘッドにばね機構を取り付けた.
            \item ばね機構に文銅をのせた.
            \item 静的な荷重でばねが変形する量を目盛尺で測定し,ばね係数を求めた.
            \item 加速度センサからの出力信号を増幅器で増幅し,FFTアナライザの時間軸波形測定モードで観測できるように結線した.
            ただし,時間軸スケール(横軸)は0.125s,電圧軸スケール(縦軸)は0.1414Vとした.
            \item 振器で振動を与え,周波数を20,40,60,80,100Hzと変化させた時の波形を観測し,記録した.
            \item その時の共振周波数からばね係数を推定した.
          \end{enumerate}
      \subsection{実験結果}
          % zikken2.tex
          % 問12参照
          40Hz が共振周波数であった.
      \subsection{考察}
        \begin{enumerate}
          \item ばね振動系の質量は何g?
            $8.43 \times 10^{-5}$g

          \item 得られた共振周波数からばね係数はいくつ?
            
            共振周波数は40Hzのときであり,ばね係数は次式で求められるので,$k_{0} = 8.43 \times 10^{-5} \times (2\pi\times40)^2 = 5.32 \times 10^3 Nm^{-1}$
            \begin{equation}
              f_{0} = \frac{\omega_{0}}{2\pi} = \frac{1}{2\pi}\sqrt{\frac{k_{0}}{m}}
            \end{equation}
            \begin{equation}
              k_{0} = m(2 \pi f_{0})^2
            \end{equation}
          \item 静的な荷重から求めたばね係数と共振周波数から推定したばね係数がどれくらい違い,なんで違うのか?
              静的な荷重$-292$g/重より,$k=\frac{4\times292\times9.8\times10^{-3}}{0.25\times4} = 1.14 \times 10^{4} Nm^{-1}$であった.
            よって,静的な荷重から求めたばね係数は共振周波数から推定したばね係数$\frac{1.14 \times 10^{4}}{5.32 \times 10^3} = 2.14$倍違った.
            異なっている理由については,共振周波数を40Hzと測定したときに20Hz毎に測定していたため,誤差が生じていたと考えられる.


          \item Aは各周波数での加振器の振動,Bは各周波数での機械振動系の振動であり,比較して何が言えるか?
            %TODO: 画像もらってからやる->問16
            Aの加振器ではそれ自身がダンパーとなるので減衰してしまうが,Bだとそれがない.つまり,現実と理想は違う.
        \end{enumerate}

    \section{1自由度振動系のインパルス応答のFFT解析}
        \subsection{目的}
            1自由度の振動機械系にインパルス入力を与えて,この時の応答の挙動を加速度センサで検出し,FFTアナライザで解析し,周波数特性を考察する.
            パワースペクトルや周波数応答関数の意味を理解する.
        \subsection{装置}
            % zikken3.tex
        \subsection{手順}
          \begin{enumerate}
            \item 加振器のガタの影響を受けないようにするため,加速度センサ付1自由度振動系を加振器から取り外し,バイスにしっかりと挟んで固定した.
            \item 機械振動系に取り付けた加速度センサの信号をアンプで増幅し,FFTアナライザのCH2に入れた.
            \item インパクトハンマーからの信号をアンプで増幅し,FFTアナライザのCH1に入れた.
            \item FFTアナライザの画面に,上部にはインパクトハンマーからの信号の時間軸応答波形,下部には振動系の加速度センサからの時間軸応答波形が表示されるようにした.
            \item FFTアナライザで波形を観測した.
            \item 伝達関数の波形を保存した
          \end{enumerate}
        \subsection{結果}
          60Hzが共振周波数だとわかった.
        \subsection{考察}
          \begin{enumerate}
            \item FFT解析結果の横軸と縦軸はそれぞれ何を表しているか
              横軸は周波数,横軸は周波数領域における入出力比である伝達関数のゲインを表している.
            \item 機械振動系において動特性を示すコンプライアンスとはどのような意味があり、機械系の動特性をどのように評価する値なのか?
              動剛性(ばね係数)の逆数であり,物体の変形のしやすさを示す値である.機械系の動特性は,コンプライアンスが小さくなる周波数では,機械系が硬くなるという性質から評価できる.
            \item 得られた結果のグラフはどのような特徴があり、これらは何を意味しているのか?特にピーク値や曲線の形について
              グラフは機械振動系に与えた周波数と振動の関係を示していた.小刻みに振動した区間において,波形も小刻みになっており,滑らかに振動した区間においては,滑らかな波形となった.
              ピーク値においては,振動系は共振していた.
            \item この結果から得られた振動系の共振周波数は前の実験から得られた結果と比べてどのような違いがあるのか
              本項で得られた共振周波数は60Hzであった.  しかし,1自由度の強制振動系の振動挙動の観察により得られた共振周波数は40Hzであったので,値に差が生じていた.これは,
              4.2の加振器自体がダンパの役割をして振動を吸収していたからだと考えられる.
          \end{enumerate}
        
      
% 結果
\begin{thebibliography}{3}
\bibitem{}電気通信大学 共通教育部自然科学部会(物理) :基礎科学実験A p73-79
\end{thebibliography}
\end{document}