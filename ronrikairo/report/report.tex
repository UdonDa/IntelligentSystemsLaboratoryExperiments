\documentclass[twocolumn, 10pt,a4j]{jsarticle}
\usepackage{amsmath}
\usepackage[dvipdfmx]{graphicx}
\usepackage{url}
% プリアンブル
\title{\vspace{-2.5cm}8. 論理回路}
\author{1610581 堀田 大地}
\date{2018/5/17}
\begin{document}
\maketitle{}
\section{目的}
% 目的
トランジスタ,IC等の半導体素子の発展と共に機械システムへのエレクトロニクスの導入が進み,
今やエレクトロニクスと関わりのない機械システムは考えられなくなった.特にコンピュータを始め,
その周辺機器,各種情報機器,NC工作機械, 家電製品等にはディジタル回路が多用されている.そこで,
実際に広く利用されているディジタル用ICを用いて,ディジタル回路,特に論理回路の基礎的事項について実験し,
ディジタルICの使い方,動作,設計法について理解する.
\section{方法}
% 方法
hoge
% \section{原理}
% 原理
% \subsection{原理-1}
% \begin{equation}
% v=\dfrac {a^{2}}{8\eta }\left( \rho g\cos \theta -\dfrac {2\gamma }{al}\right) 
% \end{equation}
% \begin{equation}
% l_{0}=\dfrac {2\gamma }{a\rho g\cos \theta }
% \end{equation}
% \subsection{原理-2}
\section{実験項目}
% 実験項目
\subsection{ゲート回路}
\subsection{2入力EX-ORゲート}
\subsection{デコーダとエンコーダ}
\subsection{加算回路}
\subsubsection{回路の機能説明}
\subsubsection{回路の動作票}
\subsubsection{回路の真理値表}
\subsubsection{回路の論理式}
\subsubsection{考察}
\subsection{ラッチ回路}
\subsection{フリップフロップ回路}
\subsection{カウンタ回路}
\subsubsection{回路の機能説明}
\subsubsection{回路の動作表}
\subsubsection{タイムチャート}
\subsubsection{考察}

\section{感想}
% 感想
hoge


\begin{thebibliography}{3}
\bibitem{}CT-311S 実習セット(デジタル編)学習の手引き,サンハヤト株式会社
\bibitem{}最新74シリーズIC規格票,CQ出版社
\bibitem{}猪飼國夫,本多中二共著,定本 ディジタルシステムの設計,CQ出版社
\end{thebibliography}
\end{document}